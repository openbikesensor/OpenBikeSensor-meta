% Options for packages loaded elsewhere
\PassOptionsToPackage{unicode}{hyperref}
\PassOptionsToPackage{hyphens}{url}
%
\documentclass{scrartcl}
\usepackage{lmodern}
\usepackage{amssymb,amsmath}
\usepackage[T1]{fontenc}
\usepackage[utf8]{inputenc}
\usepackage{textcomp} % provide euro and other symbols
\usepackage{xcolor}
\usepackage{hyperref}
\usepackage{url}
\usepackage{array}
\urlstyle{same} % disable monospaced font for URLs
\usepackage{booktabs}
% Correct order of tables after \paragraph or \subparagraph
\usepackage{etoolbox}

\usepackage{graphicx}

\renewcommand{\thesubsubsection}{\arabic{subsubsection}.}

\author{}
\date{}

\begin{document}

\section*{OpenBikeSensor Bauanleitung}

\subsection*{Bauteile:}

\begin{itemize}
	\item \textbf{HC-SR04P Sensor}
	
	Hinweis: Die Sensoren messen per Ultraschall den Abstand zum
	überholenden Fahrzeug und auch den Abstand zu parkenden Fahrzeugen. Ihr
	benötigt zwei Stück pro OBS.
	
	\href{https://www.google.com/url?q=https://www.aliexpress.com/item/33039149738.html\&sa=D\&ust=1588976963766000}{https://www.aliexpress.com/item/33039149738.html}
	
	
	\item \textbf{5-pin XS9 Aviation Connector}
	
	Hinweis: Die Push-Pull Rundsteckverbindung ist die Verbindung zwischen
	dem OBS und dem Kabel zum Push Button am Lenker. 
	
	\href{https://www.google.com/url?q=https://www.aliexpress.com/item/32512693653.html\&sa=D\&ust=1588976963767000}{https://www.aliexpress.com/item/32512693653.html}
	
	
	
	\item \textbf{12mm Push Button}
	
	Hinweis: Dieser Button ist die Drucktaste am Lenker mit dem jeder echte
	Überholvorgang eines Fahrzeugs bestätigt werden soll.
	
	\href{https://www.google.com/url?q=https://www.aliexpress.com/item/4000295670163.html\&sa=D\&ust=1588976963768000}{https://www.aliexpress.com/item/4000295670163.html}
	
	
	
	\item \textbf{0.96 inch OLED Display}
	
	Hinweis: Das Display am Lenker zeigt euch den Überholabstand in
	Zentimeter an. Das Display in dieser Form ist nicht wasserfest. Bei
	Regen bitte Folie über das Display kleben!
	
	\href{https://www.google.com/url?q=https://www.aliexpress.com/item/32896971385.html\&sa=D\&ust=1588976963769000}{https://www.aliexpress.com/item/32896971385.html}
	
	
	
	\item \textbf{18650-LiFePo Battery}
	
	Hinweis: Der Akku für den OBS. Nach letzten Messungen hält der Akku gut
	einen Tag. 
	
	\href{https://www.google.com/url?q=https://www.akkuteile.de/lifepo-akkus/18650/a123-apr18650m-a1-1100mah-3-2v-3-3v-lifepo4-akku/a-1006861/\&sa=D\&ust=1588976963770000}{https://www.akkuteile.de/lifepo-akkus/18650/a123-apr18650m-a1-1100mah-3-2v-3-3v-lifepo4-akku/a-1006861/}
	
	
	
	\item \textbf{TP5000 LiFePo-Charger}
	
	Hinweis: Lademodul mit Micro USB Anschluss. 
	
	\href{https://www.google.com/url?q=https://www.ebay.de/itm/122164745507\&sa=D\&ust=1588976963771000}{https://www.ebay.de/itm/122164745507}
	
	\href{https://www.google.com/url?q=https://www.aliexpress.com/item/4000310107151.html\&sa=D\&ust=1588976963771000}{https://www.aliexpress.com/item/4000310107151.html}
	
	
	
	\item \textbf{USB-C Lademodul}
	
	\href{https://www.google.com/url?q=https://www.ebay.de/itm/173893903484\&sa=D\&ust=1588976963772000}{https://www.ebay.de/itm/173893903484}
	
	
	
	\item \textbf{LiFePo Protection Board}
	
	\href{https://www.google.com/url?q=https://www.ebay.de/i/202033076322?ul_noapp\%3Dtrue\&sa=D\&ust=1588976963773000}{https://www.ebay.de/i/202033076322?ul\_noapp=true}
	
	
	
	\item \textbf{GPS-Modul}
	
	GYGPS6MV2 GPS Module Mini Antenna
	
	\href{https://www.google.com/url?q=https://www.ebay.de/itm/GPS-NEO-6M-7M-8M-GY-GPS6MV2-Module-Aircraft-Flight-Controller-For-Arduino/272373338855\&sa=D\&ust=1588976963774000}{https://www.ebay.de/itm/GPS-NEO-6M-7M-8M-GY-GPS6MV2-Module-Aircraft-Flight-Controller-For-Arduino/272373338855}
	
	
	
	\item \textbf{ESP32}
	
	\href{https://www.google.com/url?q=https://www.az-delivery.de/collections/bestseller/products/esp32-developmentboard\&sa=D\&ust=1588976963775000}{https://www.az-delivery.de/collections/bestseller/products/esp32-developmentboard}

\end{itemize}

%\subsection*{Inhaltsverzeichnis}

\subsection*{Materialliste}

Kabel: jetzige Version, alles 0,25\,mm${}^{2}$

{}

\protect\hypertarget{t.9a815b74fcc604715e7feb03f2d81a08b49f1f51}{}{}\protect\hypertarget{t.0}{}{}

\begin{tabular}{@{}>{\raggedright\arraybackslash}p{0.22\columnwidth}>{\raggedright\arraybackslash}p{0.22\columnwidth}>{\raggedright\arraybackslash}p{0.22\columnwidth}>{\raggedright\arraybackslash}p{0.22\columnwidth}@{}}
	\toprule
	Bauteil & ESP32 & Masse und VCC & Sonstige \\
	\midrule
	Ultraschallsensoren & 4$\times$ 12\,cm & {2$\times$ 12\,cm} &  direkte Verbindung zwischen Sensoren: 2$\times$ 3,5\,cm \\[2.8em]
	SD & 4$\times$ 10\,cm & 2$\times$ 10\,cm & direkte Verbindung Masse: 1$\times$ 2,5\,cm\\[1.8em]
	GPS & 2$\times$ 9,5\,cm & 2$\times$ 9,5\,cm &  \\[1.5em]
	ESP-VCC und GND\ &       &                   & 2$\times$ 8\,cm\\[1.8em]
	Buchse & 3$\times$ 13\,cm & 2$\times$ 13\,cm &               \\[1.5em]
	Schalter &                &                  & 2$\times$ 6\,cm\\[1.8em]
	Batterie-Schutzmodul &    &                  & 2$\times$ 2,5\,cm\\[1.8em]
	Schutzmodul-Schalter / Masse &  &            & 2$\times$ 9\,cm\\[3.0em]
	Lademodul USB-C zu LiFePo-Lader & &          & 2$\times$ 3\,cm\\[2.1em]
	Stecker-Display &                 &          & 65\,cm, 5\,pol, 5\,mm${}^{2}$\\
	\bottomrule
\end{tabular}
\clearpage


\subsection*{Baugruppen}
\begin{itemize}
	\item Ultraschallsensoren
	\item GPS
	\item SD-Karte
	\item Lademodule
	\item Batterie mit Schutzschaltung und Schalter
	\item Stecker
	\item Display
\end{itemize}

\subsection*{Endmontage}

\subsubsection{Verlöten der SD-Karte}


\begin{center}
	\includegraphics[width=1\columnwidth]{images/image4.jpg}
\end{center}

{Die SD-Karte wird fast wie in diesem Beispiel angeschlossen:}

{\href{https://www.google.com/url?q=https://camo.githubusercontent.com/fe6b89251ae4df2628b1a4c86c57976f22d6d5ba/687474703a2f2f692e696d6775722e636f6d2f34436f584f75522e706e67\&sa=D\&ust=1588976963798000}{https://camo.githubusercontent.com/fe6b89251ae4df2628b1a4c86c57976f22d6d5ba/687474703a2f2f692e696d6775722e636f6d2f34436f584f75522e706e67}}


Nur VCC und GND werden nicht direkt am ESP32 angeschlossen.


\begin{center}
	\begin{tabular}{@{}lll@{}}
		\toprule
		{Bezeichnung} & {Farbe} & {ESP32 Pin}\\
		\midrule
		{MISO} & {Lila} & {19}\\
		{GND} & {Schwarz} & {}\\
		{CLK} & {Rot} & {18}\\
		{VCC} & {Blau} & {}\\
		{GND} & {Schwarz} & {}\\
		{MOSI} & {Rosa} & {23}\\
		{CS} & {Grau} & {5}\\
		\bottomrule
	\end{tabular}
\end{center}


\subsubsection{GPS-Modul}

\begin{center}
	\includegraphics[width=\columnwidth]{images/image9.jpg}
\end{center}


\begin{center}
	\begin{tabular}{@{}lll@{}}
		\toprule
		{Bezeichnung} & {Farbe} & {ESP32 Pin}\\
		\midrule
		{VCC} & {Gelb} & {-\/-\/-}\\
		{RX} & {Grün} & {17}\\
		{TX} & {Braun} & {16}\\
		{GND} & {Weiß} & {-\/-\/-}\\
		\bottomrule
	\end{tabular}
\end{center}

\subsubsection{Ultraschallsensoren}

\begin{center}
	\begin{tabular}{@{}lll@{}}
		\toprule
		{Sensor am Deckel} & {Farbe} & {ESP32 Pin}\\
		\midrule
		{VCC} & {Rot - kurze Brücke zu anderem Sensor} &
		{-\/-\/-}\\
		{Trig} & {Grün} & {15}\\
		{Echo} & {Orange} & {4}\\
		{GND} & {Schwarz - kurze Brücke zu anderem Sensor} &
		{-\/-\/-}\\
		\bottomrule
	\end{tabular}
\end{center}

\begin{center}
	\begin{tabular}{@{}lll@{}}
		\toprule
		{Sensor im Gehäuse} & {Farbe} & {ESP32 Pin}\\
		\midrule
		{VCC} & {Rot} & {-\/-\/-}\\
		{Trig} & {Grün} & {25}\\
		{Echo} & {Orange} & {26}\\
		{GND} & {Schwarz} & {-\/-\/-}\\
		\bottomrule
	\end{tabular}
\end{center}

\subsubsection{Stecker zum Display}

\begin{center}
	\begin{tabular}{@{}lll@{}}
		\toprule
		{Bezeichnung} & {Stecker Pin} & {ESP32 Pin}\\
		\midrule
		{VCC} & {1} & {-\/-\/-}\\
		{SCL} & {2} & {22}\\
		{Druckknopf} & {3} & {2}\\
		{GND} & {4} & {-\/-\/-}\\
		{SDA} & {5} & {21}\\
		\bottomrule
	\end{tabular}
\end{center}


Druckknopf und Display teilen sich VCC. Der Pin am ESP32, an dem der Schalter hängt wird auch mit 10kOhm-Widerstand mit GND verbunden.


\begin{center}
	\includegraphics[width=0.5\columnwidth]{images/image3.jpg}
\end{center}

Die Kappe des Steckers ist raus zu schrauben. Die Außenisolierung des
Kabels sollte nur sehr knapp entfernt werden um die Zugentlastung
nachträglich fest schrauben zu können. (Zange zum Gegenhalten nur zur
Verdeutlichung, besser verwendet man den eingesteckten Stecker als
Gegenhalt.)


\includegraphics[width=\columnwidth]{images/image14.jpg}\\

\includegraphics[width=\columnwidth]{images/image20.jpg}\\

\includegraphics[width=\columnwidth]{images/image8.jpg}\\

\includegraphics[width=\columnwidth]{images/image16.jpg}\\

\includegraphics[width=\columnwidth]{images/image18.jpg}\\

\includegraphics[width=\columnwidth]{images/image1.jpg}\\

\includegraphics[width=\columnwidth]{images/image5.jpg}\\

\includegraphics[width=\columnwidth]{images/image19.jpg}\\

\includegraphics[width=\columnwidth]{images/image6.jpg}\\

\includegraphics[width=\columnwidth]{images/image21.jpg}\\

\includegraphics[width=\columnwidth]{images/image2.jpg}\\

\includegraphics[width=\columnwidth]{images/image17.jpg}\\

\includegraphics[width=\columnwidth]{images/image11.jpg}\\

\includegraphics[width=\columnwidth]{images/image15.jpg}\\

\includegraphics[width=\columnwidth]{images/image12.jpg}\\

\includegraphics[width=\columnwidth]{images/image7.jpg}\\

\includegraphics[width=\columnwidth]{images/image13.jpg}\\

\includegraphics[width=\columnwidth]{images/image10.jpg}\\

\end{document}
